Vogliamo aggiungere al nucleo il meccanismo delle interruzioni inter-processo.
Un qualunque processo pu\`o inviare una interruzione ad un altro processo di cui conosce l'identificatore.
L'interruzione setta un flag nel descrittore del processo destinatario. Ogni processo pu\`o esaminare
lo stato del proprio flag con una opportuna primitiva.

Se il processo destinatario si era sospeso sulla coda del timer con la primitiva \verb|delay(natl n)|,
l'operazione di interruzione, invece di settare il flag, risveglia il processo. La primitiva
\verb|delay(natl n)| pu\`o ora terminare per due motivi distinti: o perch\'e sono trascorsi \verb|n|
intervalli di tempo, o perch\'e il processo \`e stato interrotto. Per poter distingure questi due casi
la primitiva restituisce un valore al chiamante. Tale valore \`e il numero di intervalli di tempo
che il processo avrebbe dovuto ancora attendere (e dunque \`e 0 se il
processo si \`e risvegliato normalmente).

A tale scopo aggiungiamo i seguenti campi al descrittore di ogni processo:
\begin{verbatim}
        bool  interrupted;
        bool  sleeping;
\end{verbatim}
Il campo \verb|interrupted| \`e il flag di interruzione, posto a \verb|false| alla creazione del processo.
Il campo \verb|sleeping| \`e un flag che vale \verb|true| se il processo si trova nella coda del timer,
e \verb|false| altrimenti.

Modifichiamo inoltre la primitiva \verb|delay| in modo che restituisca un \verb|natl|,
e le funzioni \verb|c_delay| e \verb|c_driver_td| in modo che gestiscano
il flag \verb|sleeping| e il valore di ritorno della \verb|delay|
(per le modifiche a queste due funzioni si rimanda al file \verb|sistema.cpp| fornito)

Aggiungiamo infine le seguenti primitive:
\begin{itemize}
\item \verb|bool interrupt(natl id)|:
  Se \verb|id| non corrisponde ad un processo esistente restituisce \verb|false| e termina.
  In tutti gli altri casi restituisce \verb|true|.
  Se il flag \verb|interrupted| \`e gi\`a a \verb|true| non fa altro.
  Se il flag \verb|interrupted| \`e \verb|false| e il processo non \`e nella coda del timer, si limita a settare
  l'opportuno flag.
  Infine, se il flag \verb|interrupted| \`e false ed il processo si trova nella coda del timer
  lo risveglia. Per risvegliarlo deve rimuoverlo dalla coda del timer, avendo cura
  di riportarla in uno stato consistente.
  Aggiorna opportunamente i campi del descrittore di processo 
  e gestisce eventuali {\em preemption}.
\item \verb|bool interrupted()| (gi\`a realizzata):
  restituisce il valore del flag \verb|interrupt| del processo corrente e lo resetta.
\end{itemize}

Modificare i file \verb|sistema.cpp| e \verb|sistema.s| in modo da
realizzare la primitiva \verb|interrupt()|.
