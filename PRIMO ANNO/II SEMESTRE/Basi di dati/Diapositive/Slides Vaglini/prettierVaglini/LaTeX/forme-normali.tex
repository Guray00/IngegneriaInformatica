\documentclass{beamer}
\usetheme{Boadilla}
\usepackage[utf8]{inputenc}
\usepackage{amsmath}
\usepackage{fixltx2e}
\usepackage{hyperref}
\usepackage{listings}
\usepackage{multicol}

\AtBeginSection[]{
  \begin{frame}
  \vfill
  \centering
  \begin{beamercolorbox}[sep=8pt,center,shadow=true,rounded=true]{title}
    \usebeamerfont{title}\insertsectionhead\par%
  \end{beamercolorbox}
  \vfill
  \end{frame}
}

\definecolor{arsenic}{rgb}{0.23, 0.27, 0.29}
\definecolor{cadet}{rgb}{0.33, 0.41, 0.47}

\title{Le forme normali 1/2}
\institute{Università di Pisa}
\date{Lezione del 14 aprile 2021}
% \logo{\includegraphics[scale=.25]{blue-unipi.png}}


\begin{document}

\frame{\titlepage}
\begin{frame}{Contenuti}
\begin{multicols}{2}
  \tableofcontents
\end{multicols}
\end{frame}


% ------------------------------------------ INTRODUZIONE ------------------------------------------ %
\section{Introduzione}

\begin{frame}{Obbiettivi}
    
    \begin{itemize}
        \item[$\blacktriangleright$] valutare la qualità della progettazione degli schemi relazionali 
        \item[$\blacktriangleright$]  conservazione dell’informazione
        \item[$\blacktriangleright$]  minimizzazione della ridondanza
    \end{itemize}
    \vfill
    Approccio top-down: raggruppamenti di attributi analizzati e successivamente decomposti.
    
\end{frame}

\begin{frame}{Linee guida}

    \begin{enumerate}
        \item \textbf{Semplice è bello} \\ Non raggruppare attributi da più tipi di entità/relazione in un’unica relazione. 
        \item \textbf{No alle anomalie} \\  Certificare che i programmi di inserimento, cancellazione o modifica funzionano sempre.
        \item \textbf{Evitare valori nulli} \\  Se inevitabili, assicursi che sono pochi rispetto al numero di tuple di una relazione.
    \end{enumerate}
    
\end{frame}


\begin{frame}{Esempio: LG1}

    \begin{itemize}
    
        \item[$\blacktriangleright$] Per mantenere semplicità semanticare, far corrispondere:\\
        un schema di relazione $\rightarrow$ un solo tipo di entità
    \end{itemize}


\end{frame}


\begin{frame}{Esempio: LG2}

    \Large Anagrafe(CF, NomeP, Indirizzo, NomeC, NumAb)
    \vfill
    \normalsize Semantica attributi:
    \begin{itemize}
        \item[$\blacktriangleright$] CF determina NomeP, Indirizzo e NomeC
        \item[$\blacktriangleright$] NomeC determina NumAb
    \end{itemize}
    \vfill
    Considerazioni:
    \begin{itemize}
        \item[$\blacktriangleright$] NumAb ripetuto per tuple con lo stesso NomeC \\ $\implies$ da mantenere consistenza
        \item[$\blacktriangleright$] evitare problema trasformando lo schema in due schemi:
            \begin{itemize}
                \item[•] Persona(CF, NomeP, Indirizzo, NomeC)
                \item[•] Residenza(NomeC, NumAb)
            \end{itemize}
    \end{itemize}

\end{frame}

% ------------------------------------------ FORMALIZZAZIONE ------------------------------------------ %
\section{Formalizzazione}

\subsection{In teoria}
\begin{frame}{Dipendenza funzionale: DF}

\begin{block}{Definizione}
Esprime un legame semantico tra due gruppi di attributi di uno schema R.
\end{block}
    \begin{itemize}
        \item[$\blacktriangleright$] è una proprietà di R, non di un particolare stato valido r di R
        \item[$\blacktriangleright$] non può essere dedotta a partire da uno stato valido r
        \item[$\blacktriangleright$] deve essere definita esplicitamente da qualcuno che conosce la semantica degli attributi di R
    \end{itemize}
\end{frame}

\begin{frame}{Forme normali}

    \begin{itemize}
        \item[$\blacktriangleright$] esistono diversi tipi, ciascuno:
            \begin{itemize}
                \item[$\bullet$] garantisce l'assenza di determinati difetti in R
                \item[$\bullet$] quindi definisce un determinato livello di qualità di R
            \end{itemize}
        \item[$\blacktriangleright$] possibile eseguire una serie di test per certificare che R soddisfa una data forma normale
    \end{itemize}
\end{frame}

\begin{frame}{Normalizzazione}
    \begin{block}{Definizione}
        La \textbf{procedura} che permette di portare uno schema relazionale in una determinata forma normale si dice normalizzazione.
    \end{block}
    
    \begin{block}{Osservazione}
        La normalizzazione può essere utilizzata come \textbf{tecnica di verifica} dei risultati della progettazione, non costituisce una metodologia di progetto.
    \end{block}
\end{frame}

\subsection{In pratica}
\begin{frame}{Esempio (1/2)}
    \Large R (\underline{Impiegato}, Stipendio, \underline{Progetto}, Bilancio, Funzione)
    \vfill
    \normalsize Considerazioni:
    \begin{itemize}
        \item[$\blacktriangleright$] Ogni impiegato può partecipare a più progetti, sempre con lo stesso stipendio, e con una sola funzione per progetto.
        \item[$\blacktriangleright$] Ogni progetto ha un bilancio indipendentemente da quanti dipendenti ci lavorano 
    \end{itemize}
    

\end{frame}
\begin{frame}{Esempio (2/2)}
    \normalsize Possibili anomalie:
    \begin{itemize}
        \item[$\blacktriangleright$] \textbf{di aggiornamento}
            \begin{itemize}
                \item[$\bullet$] se lo stipendio di un impiegato varia $\rightarrow$ variano altre tuple (quali?)
                \item[$\bullet$] se bilancio di progetto varia $\rightarrow$ variano altre tuple (quali?)
            \end{itemize}
        \item[$\blacktriangleright$] \textbf{di cancellazione}
            \begin{itemize}
                \item[$\bullet$] se un impiegato si licenzia, dobbiamo cancellarlo in diverse ennuple
            \end{itemize}
    \end{itemize}
    
    
    \begin{block}{Causa dei problemi}
        Abbiamo ripetizione dello stipendio di un impiegato e del bilancio di un progetto.
    \end{block}
    
    \begin{alertblock}{Errore di progetto}
        Usare un'unica relazione per rappresentare gruppi di informazione eterogenee.
    \end{alertblock}
    

\end{frame} % possibilmente da aggiustare


% ------------------------------------------ DIPENDENZE FUNZIONALI ------------------------------------------ %
\section{Dipendenze funzionali}


\subsection{Definizione}
\begin{frame}{Ma quindi cosa sono le DF??}

    Siano:
    \begin{itemize}
        \item[$\bullet$] \textbf{r} relazione su \textbf{R}(\textbf{X})
        \item[$\bullet$] due sottoinsiemi non vuoti di \textbf{X} $\supseteq$ \textbf{Y}, \textbf{Z}
    \end{itemize}
    
    

    \begin{block}{Definizione}
        Esiste in \textbf{r} una \textcolor{blue}{dipendenza funzionale} da \textbf{Y} a \textbf{Z} se \\
        \centering
        $\forall t_1$, $t_2$ tuple in \textbf{r}, $t_1[\textbf{Y}] = t_2[\textbf{Y}]\implies t_1[\textbf{Z}] = t_2[\textbf{Z}]$
    \end{block}

    \vfill
    \begin{center}
        \Huge
        \textbf{Y} $\rightarrow$ \textbf{Z}
    \end{center}
    \vfill
    
    
\end{frame}

\begin{frame}{}

    
    \begin{alertblock}{Nota bene!}
        \centering
        \textbf{Y} $\rightarrow$ \textbf{Z} non implica \textbf{Z} $\rightarrow$ \textbf{Y}!
    \end{alertblock}

    \begin{block}{DF complete}
        Una dipendenza funzionale si dice \textcolor{blue}{completa} se \\
        \centering
        \textbf{Y} $\rightarrow$ \textbf{Z}, e $\forall \textbf{W} \subseteq \textbf{Y}$, non vale \textbf{W} $\rightarrow$ \textbf{Y}
    \end{block}
    
    \begin{itemize}
        \item[$\blacktriangleright$] Se \textbf{Y} superchiave di \textbf{R}(\textbf{X}), allora \textbf{Y} determina ogni altro attributo della relazione $\implies$ \textbf{Y} $\rightarrow$ \textbf{X}
        \item[$\blacktriangleright$] Se \textbf{Y} chiave, allora \textbf{Y} $\rightarrow$ \textbf{X} è una DF completa
    \end{itemize}
    
\end{frame}

\subsection{Esempio}
\begin{frame}{Esempio}
    \Large R (\underline{Impiegato}, Stipendio, \underline{Progetto}, Bilancio, Funzione)
    \vfill
    \normalsize Caratterizziamo le dipendenze funzionali:
    \begin{itemize}
        \item[$\blacktriangleright$] Impiegato $\rightarrow$ Stipendio
        \item[$\blacktriangleright$] Progetto $\rightarrow$ Bilancio
        \item[$\blacktriangleright$] Impiegato Progetto $\rightarrow$ Funzione
    \end{itemize}

\begin{block}{Definizione}
    Una DF \textbf{Y} $\rightarrow$ \textbf{Z} si dice \textcolor{blue}{banale} se $\forall \;$Z\textsubscript{i} attributo di \textbf{Z}, vale \textbf{Y} $\rightarrow$ Z\textsubscript{i}
\end{block}
    
    \begin{itemize}
        \item[$\blacktriangleright$] Impiegato Progetto $\rightarrow$ Progetto
            (DF banale, sempre soddisfatta.)
    \end{itemize}

    
\end{frame}

\begin{frame}{Legame fra DF e anomalie}
    Notare che:
    \begin{itemize}
        \item[$\blacktriangleright$] Impiegato $\rightarrow$ Stipendio, ci sono ripetizioni
        \item[$\blacktriangleright$] Progetto $\rightarrow$ Bilancio, ci sono ripetizioni
        \item[$\blacktriangleright$] Impiegato Progetto $\rightarrow$ Funzione, \textbf{non} ci sono ripetizioni
    \end{itemize}
    \vfill
    \Large Le informazioni legate ad attributi non chiave\par causano problemi.
\end{frame}

\subsection{Teoria sulle dipendenze}
\begin{frame}{Implicazione}
    Siano F un insieme di dipendenze funzionali su \textbf{R}(\textbf{Z}) e \textbf{X} $\rightarrow$ \textbf{Y}\par
    Allora:
    \begin{block}{Definizione}
        Se $\forall \;$\textbf{r} istanza di \textbf{R} che verifica tutte le dipendenze in F, risulta verificata anche \textbf{X} $\rightarrow$ \textbf{Y}, allora si dice che F implica \textbf{X} $\rightarrow$ \textbf{Y}.
    \end{block}
    \vfill
    \centering
    \Huge F $\implies$ \textbf{X} $\rightarrow$ \textbf{Y}
    \vfill
\end{frame}

\begin{frame}{Chiusura}
    Siano F un insieme di dipendenze funzionali su \textbf{R}(\textbf{Z}) e \textbf{X} $\rightarrow$ \textbf{Y}, allora la chiusura di F è l'insieme di tutte le dipendenze funzionali implicate da F: 
    \begin{block}{Definizione}
        \centering
        F\textsuperscript{+} = \{ \textbf{X} $\rightarrow$ \textbf{Y} $|$ F $\implies$ \textbf{X} $\rightarrow$ \textbf{Y} \}
    \end{block}
    
    \begin{alertblock}{Osservazione}
        Se un'istanza \textbf{r} dello schema \textbf{R} soddisfa F, allora soddisfa anche F\textsuperscript{+}.
    \end{alertblock}
\end{frame}

\begin{frame}{Superchiave}
    Siano F un insieme di dipendenze funzionali su \textbf{R}(\textbf{Z}), e \textbf{K} $\subseteq$ \textbf{Z}. Allora:
    \begin{block}{Definizione}
         \textbf{K} si dice superchiave di R se la dipendenza funzionale \textbf{K} $\rightarrow$ \textbf{Z} è \textcolor{blue}{logicamente implicata} da F, ovvero se \textbf{K} $\rightarrow$ \textbf{Z} $\in$ F\textsuperscript{+}.
    \end{block}
    
    \vfill
    Ricordiamo che se nessun insieme proprio di \textbf{K} è superchiave di \textbf{R}, allora \textbf{K} si dice chiave di \textbf{R}. Gli attributi di una chiave non si possono ottenere da nessuna DF, si deve partire da loro per ottenere gli altri.

\end{frame}



% ------------------------------------------ COSTRUZIONE DELLA CHIUSURA ------------------------------------------ %
\section{Costruzione della chiusura}
\subsection{Motivazione}
\begin{frame}{Problema}
    \begin{block}{}
        Trovare tutte le chiavi di una relazione --- al peggio esponenziale. \par
        Rinuncia alla "chiusura totale"
    \end{block}
    \vfill
    Algoritmo:
    \begin{itemize}
        \item[$\bullet$] attributi SOLO a sx delle DF stanno in tutte le chiavi
        \item[$\bullet$] chiamo l'insieme di quei attributi N
        \item[$\bullet$] calcola N\textsuperscript{+}
        \item[$\bullet$] aggiungi ad N un attributo alla volta, poi una coppia, etc
    \end{itemize}
\end{frame}

\begin{frame}{Calcolo di F\textsuperscript{+}}

    La definizione di \textcolor{blue}{implicazione} prevede un \textcolor{blue}{quantificatore universale}.
    \begin{itemize}
        \item[$\blacktriangleright$] non utilizzabile in pratica
        \item[$\blacktriangleright$] servono \textcolor{blue}{regole di inferenza} per derivare costruttivamente le DF
    \end{itemize}
    \vfill
    1974, Armstrong \href{https://web.archive.org/web/20180126091352if_/https://ipfs.io/ipfs/QmWYWTGUZyTm2iRFTZY2pTr2x1vWkDiJr2CBp2PGVpSVSv}{"Dependency structures of data base relationships"}\par
    (Spiegati bene anche su \href{https://it.wikipedia.org/wiki/Assiomi_di_Armstrong}{Wikipedia})
    
    \vfill
\end{frame}

\subsection{Regole di inferenza di Armstrong}
\begin{frame}{Regole di inferenza di Armstrong}
    \begin{enumerate}
        \item \textbf{Riflessività}: se \textbf{Y} $\subseteq$ \textbf{X}, allora \textbf{X} $\rightarrow$ \textbf{Y}
        \item \textbf{Additività/Arricchimento}: se \textbf{X} $\rightarrow$ \textbf{Y}, allora \textbf{XZ} $\rightarrow$ \textbf{YZ} $\forall$\textbf{Z}
        \item \textbf{Transitività}: se \textbf{X} $\rightarrow$ \textbf{Y} e \textbf{Y} $\rightarrow$ \textbf{Z}, allora \textbf{X} $\rightarrow$ \textbf{Z}
    \end{enumerate}
\end{frame}

\begin{frame}{Proprietà}
    \begin{block}{Teorema correttezza}
    Applicandole ad un insieme F di dipendenze funzionali, si ottengono solo dipendenze logicamente implicate da F.
    \end{block}
    
    \begin{block}{Teorema completezza}
    Applicandole ad un insieme F di dipendenze funzionali, si ottengono tutte le dipendenze logicamente implicate da F.
    \end{block}
    
    \begin{block}{Teorema minimalità}
    Ignorando anche una sola regola, l’insieme di regole che rimangono non è più completo.
    \end{block}
\end{frame}

\begin{frame}{Dimmostrazione addittività: per assurdo}
    Supponiamo che $\exists \;$\textbf{r} istanza di \textbf{R} $|$ vale \textbf{X} $\rightarrow$ \textbf{Y}, ma non \textbf{XZ} $\rightarrow$ \textbf{YZ} \par
    Allora $\exists \; t_1, t_2$ di \textbf{r} tali che:
    \begin{enumerate}
        \item[(1)] $t_1$[\textbf{X}] = $t_2$[\textbf{X}]
        \item[(2)] $t_1$[\textbf{Y}] = $t_2$[\textbf{Y}]
        \item[(3)] $t_1$[\textbf{XZ}] = $t_2$[\textbf{XZ}]
        \item[(4)] $t_1$[\textbf{YZ}] $\neq t_2$[\textbf{YZ}]
    \end{enumerate}
    \vfill
    Da (1) e (3) si deduce: \textcolor{blue}{(5)} $t_1$[\textbf{Z}] = $t_2$[\textbf{Z}] \par
    Mentre da (2) e (5) abbiamo: \textcolor{blue}{(6)} $t_1$[\textbf{YZ}] = $t_2$[\textbf{YZ}],\par
    in contradizione con (4), assurdo.\par
    \raggedleft
    \huge
    $\blacksquare$\hspace{1cm}
\end{frame}

\begin{frame}{Dimmostrazione transitività: sempre per assurdo}
    Fai finta che $\exists \;$\textbf{r} istanza di \textbf{R} $|$ valgono \textbf{X} $\rightarrow$ \textbf{Y} e \textbf{Y} $\rightarrow$ \textbf{Z}, ma non \textbf{X} $\rightarrow$ \textbf{Z} \par
    Allora $\exists \; t_1, t_2$ di \textbf{r} tali che:
    \begin{enumerate}
        \item[(1)] $t_1$[\textbf{X}] = $t_2$[\textbf{X}]
        \item[(2)] $t_1$[\textbf{Y}] = $t_2$[\textbf{Y}]
        \item[(3)] $t_1$[\textbf{Z}] = $t_2$[\textbf{Z}], ma anche
        \item[(4)] $t_1$[\textbf{Z}] $\neq t_2$[\textbf{Z}]
    \end{enumerate}
    in contradizione con (3), assurdo.\par
    \raggedleft\huge
    $\blacksquare$\hspace{1cm}
    
\end{frame}


\begin{frame}{Dimmostrazione riflessività: fatta io, aperta a critiche}
    Supponiamo che $\exists \;$\textbf{Y} $\subseteq$ \textbf{X} $|$ non vale \textbf{X} $\rightarrow$ \textbf{Y}.\par 
    Comunque deve valere \textbf{X} $\rightarrow$ \textbf{X}, DF banale. Allora $\exists \; t_1, t_2$ di \textbf{r} tali che:
    \begin{enumerate}
        \item[(1)] $t_1$[\textbf{X}] = $t_2$[\textbf{X}]. \\
        Ma per ipotesi \textbf{Y} $\subseteq$ \textbf{X}, quindi per forza di cose vale:
        \item[(2)] $t_1$[\textbf{X}] = $t_2$[\textbf{Y}],
    \end{enumerate}
    in contradizione con l'ipotesi che non vale \textbf{X} $\rightarrow$ \textbf{Y}, assurdo.\par
    \raggedleft
    \huge
    $\blacksquare$\hspace{1cm}
\end{frame}

\begin{frame}[fragile]{Algoritmo per trovare F\textsuperscript{+}}
    \begin{verbatim}
    F+ = F
    ripeti
        per ogni DF f in F+
            aggiungi riflessivita(f) ad F+
            aggiungi arricchimento(f) ad F+
        fine
        per ogni DF f1, f2 in F+
            // f1 = X->Y, f2 = Z->W
            se Y == Z
                aggiungi transitivita(f1, f2) ad F+
        fine
    fine
        \end{verbatim}
    \begin{itemize}
        \item[$\blacktriangleright$] complessità esponenziale: il per ogni nasconde il problema di trovare tutti i sottoinsiemi (backtracking), che va come $2^n$.
    \end{itemize}
\end{frame}

% \begin{frame}[fragile]{Algoritmo per trovare F\textsuperscript{+} --- ok, c'ho provato}
%     \tiny
%     \begin{verbatim}
% vector<DF> F+ = F;
% vector<DF> oldF+ = NULL;
% vector<Attributi> A = tutti gli attributi dello schema;

% while(equal(oldF+, F+))
% {
%     oldF+ = F+;
    
%     // applica riflessivita
%     for (DF f : F+)
%         // f = X → Y, Y incluso in X
%         for (Attribute Yi : Y)
%             F+.push_back(makeDF(X, Yi));
    
%     // applica arricchimento
%     for (DF f : F+)
%         // f = X → Y, Z incluso in A
%         for (Attribute Z : A)
%             F+.push_back(makeDF(XZ, YZ));
    
%     // applica transitivita
%     for (DF f1 : F+)
%         for (DF f2 : F+)
%             // f1 = X → Y, f2 = Y → Z
%             for (Attribute Zi : Z)
%             {
%                 f3 = transitivity(f1, makeDF(Y, Zi));
%                 if (f3 != NULL)
%                     F+.push_back(f3);
%             }
% }
%     \end{verbatim}
% \end{frame}

\begin{frame}{Esempio}
    Sia lo schema \textbf{R} (A, B, C, G, H, I), con le dipendenze funzionali\par
    F = 
    \{
        A $\rightarrow$ B,
        A $\rightarrow$ C,
        CG $\rightarrow$ H,
        CG $\rightarrow$ I,
        B $\rightarrow$ H
    \}
    Alcuni membri di F\textsuperscript{+} sono:
    \begin{itemize}
        \item[$\bullet$] A $\rightarrow$ H, ottenuto da Transitività(A $\rightarrow$ B, B $\rightarrow$ H);
        \item[$\bullet$] AG $\rightarrow$ I, ottenuto da Arricchimento(A $\rightarrow$ C, G) = AG $\rightarrow$ CG
        \\ \hspace{3cm} e Transitività(AG $\rightarrow$ CG, CG $\rightarrow$ I);
        \item[$\bullet$] CG $\rightarrow$ HI, ottenuto da Arricchimento(CG $\rightarrow$ I, CG) = CG $\rightarrow$ CGI
        \\ \hspace{3cm} e Arricchimento(CG $\rightarrow$ H, I) = CGI $\rightarrow$ HI
        \\ \hspace{3cm} e Transitività(CG $\rightarrow$ CGI, CGI $\rightarrow$ HI);
    \end{itemize}
\end{frame}

\subsection{Regole derivate di Armstrong}
\begin{frame}{Regole derivate di Armstrong}
    \begin{enumerate}
    \item \textbf{Unione}: \{\textbf{X} $\rightarrow$ \textbf{Y}, \textbf{X} $\rightarrow$ \textbf{Z}\} $\implies$ \textbf{X} $\rightarrow$ \textbf{YZ}
    \item \textbf{Pseudotransitività}: \{\textbf{X} $\rightarrow$ \textbf{Y}, \textbf{WY} $\rightarrow$ \textbf{Z}\} $\implies$ \textbf{WX} $\rightarrow$ \textbf{Z}
    \item \textbf{Decomposizione}: se \textbf{Z} $\subseteq$ \textbf{Y}, \{\textbf{X} $\rightarrow$ \textbf{Y}\} $\implies$ \textbf{X} $\rightarrow$ \textbf{Z}
    \end{enumerate}
\end{frame}



\begin{frame}{Dimostrazione della Regola dell'Unione}
    Le regole derivate si dimostrando a partire dalle regole di braccio forte. \par
    Infatti, per ipotesi valgono:
    \begin{enumerate}
        \item[(1)] \textbf{X} $\rightarrow$ \textbf{Y}, che per addittività diventa \textcolor{blue}{(3)} \textbf{XZ} $\rightarrow$ \textbf{YZ}
        \item[(2)] \textbf{Y} $\rightarrow$ \textbf{Z}, che per addittività diventa \textcolor{blue}{(4)} \textbf{XX} $\rightarrow$ \textbf{XZ} = \textbf{X} $\rightarrow$ \textbf{XZ}
    \end{enumerate}
    
    Applicando la transitività a (4) e (3), segue la tesi. \par
    \raggedleft
    \huge
    $\blacksquare$\hspace{1cm}
    
\end{frame}

\begin{frame}{Esempio (LG2)}
    \Large R (\underline{Impiegato}, Stipendio, \underline{Progetto}, Bilancio, Funzione)
    \normalsize
    \begin{itemize}
        \item[$\blacktriangleright$] Ogni impiegato può partecipare a più progetti, sempre con lo stesso stipendio, e con una sola funzione per progetto.
        \item[$\blacktriangleright$] Ogni progetto ha un bilancio indipendentemente da quanti dipendenti ci lavorano 
    \end{itemize}
    \hrulefill
    \begin{enumerate}
        \item[(a)] Impiegato $\rightarrow$ Stipendio \textcolor{blue}{+} arricchimento
        \\ \hspace{2.5cm}\textcolor{blue}{=} Impiegato Proggetto $\rightarrow$ Stipendio Proggetto \textcolor{red}{(d)}
        \item[(b)] Progetto $\rightarrow$ Bilancio \textcolor{blue}{+} arricchimento
        \\ \hspace{2.5cm}\textcolor{blue}{=} Proggetto Impiegato $\rightarrow$ Stipendio Impiegato \textcolor{red}{(e)}
        \item[(c)] Impiegato Progetto $\rightarrow$ Funzione
    \end{enumerate}
    Applicandocenter la Regola dell'Unione alle \{(c), \textcolor{red}{(d)}, \textcolor{red}{(e)}\} si trova:\par 
    \begin{center}
        \colorbox{pink}{Impiegato Proggetto $\rightarrow$ Stipendio Proggetto Impiegato Funzione,} 
    \end{center}
    e quindi {Impiegato, Proggetto} è chiave.
    
\end{frame}

% ------------------------------------------ EQUIVALENZA FRA INSIEMI DI DF ------------------------------------------ %
\section{Equivalenza fra insiemi di DF}
\subsection{Il problema dell'equivalenza}
\begin{frame}{Il problema dell'equivalenza}
    Dati F, G insiemi di DF, torna utile sapere se sono \textcolor{blue}{equivalenti}. Ovvero:
    \begin{block}{Definizione}
        F, G si dicono equivalenti se F\textsuperscript{+} = G\textsuperscript{+}, quindi 
        
        \begin{itemize}
            \item[$\bullet$] $\forall$ \textbf{X} $\rightarrow$ \textbf{Y} $\in$ F risulta \textbf{X} $\rightarrow$ \textbf{Y} $\in$ G\textsuperscript{+}, e viceversa
            \item[$\bullet$] $\forall$ \textbf{X} $\rightarrow$ \textbf{Y} $\in$ G risulta \textbf{X} $\rightarrow$ \textbf{Y} $\in$ F\textsuperscript.
        \end{itemize}
        
    \end{block}
    
    \vfill
    \begin{itemize}
        \item[$\blacktriangleright$] In pratica, per verificare l'equivalenza, prendi tutte le DF in F, e prova a dimostrarle usando le DF in G, e viceversa. Se tutto va bene, congratulazioni, F equivalente a G!
    \end{itemize}
\end{frame}


\subsection{La chiusura di un insieme di attributi}
\begin{frame}{Chiusura transitiva di un insieme di attributi}
    Il calcolo di F\textsuperscript{+} è molto costoso, ma tutte le DF banali manco servono! Spesso ci interessa se F\textsuperscript{+} contiene qualche specifica dipendenza. Che fare?
    \begin{block}{Definizione}
    La chiusura di un insieme di attributi A è l'insieme di tutti gli attributi che dipendono da A, relativamente a un dato F.
    \end{block}
    
    \vfill
    Boom!
    \begin{alertblock}{Teorema}
        \centering
        \textbf{X} $\rightarrow$ \textbf{Y} $\in$ F\textsuperscript{+} $\iff$ \textbf{Y} $\subseteq$ \textbf{X}\textsuperscript{+}
    \end{alertblock}
    Quindi è sufficiente fare la chiusura di \textbf{X} per capire se la dipendenza vale
\end{frame}

\begin{frame}[fragile]{Algoritmo per trovare \textbf{X}\textsuperscript{+}}

    \begin{verbatim}
    InsiemeAttributi X+ = X
    InsiemeAttributi oldX+ = NULL
    InsiemeDF F
    
    finche (X+ != oldX+)
    ripeti
        oldX+ = X+
        per ogni DF Vi -> Wi in F
            se (Vi incluso in X+) e (Wi non incluso in X+)
                X+ = X+ reunito con Wi
            fine
        fine
    fine
    \end{verbatim}
\end{frame}

\begin{frame}{Esempio: calcola se A è superchiave}
Sia F = \{A $\rightarrow$ B, BC $\rightarrow$ D, B $\rightarrow$ E, E $\rightarrow$ C\}. Calcoliamo A\textsuperscript{+}
\begin{itemize}
    \item[$\bullet$] A\textsuperscript{+} = A
    \item[$\bullet$] A\textsuperscript{+} = AB poiché A $\rightarrow$ B e A $\subseteq$ A\textsuperscript{+}
    \item[$\bullet$] A\textsuperscript{+} = ABE poiché B $\rightarrow$ E e B $\subseteq$ A\textsuperscript{+}
    \item[$\bullet$] A\textsuperscript{+} = ABEC poiché E $\rightarrow$ C e E $\subseteq$ A\textsuperscript{+}
    \item[$\bullet$] A\textsuperscript{+} = ABCED poiché BC $\rightarrow$ D e BC $\subseteq$ A\textsuperscript{+}
\end{itemize}
\begin{block}{}
    Quindi da A dipendono tutti gli altri attributi, ovvero A è superchiave \par(e anche chiave)!
\end{block}
\end{frame}

\begin{frame}{Esempio: calcola se F,G equivalenti}
Siano: F = \{A $\rightarrow$ C, AC $\rightarrow$ D, E $\rightarrow$ AD, E $\rightarrow$ H\},\par
\hspace{1cm} G = \{A $\rightarrow$ CD, E $\rightarrow$ AH\}

\begin{block}{}
    Invece di verificare se $\forall$ \textbf{X} $\rightarrow$ \textbf{Y} $\in$ F anche in G\textsuperscript{+}, vedo se \textcolor{blue}{\textbf{Y} $\subseteq$ (\textbf{X})\textsuperscript{+G}}\par (notazione fancy per chiusura di \textbf{X} rispetto a G) e viceversa.
\end{block}

\begin{itemize}
    \item[$\bullet$] per A $\rightarrow$ C si ha (A)\textsuperscript{+G} = ACD, C $\subseteq$ (A)\textsuperscript{+G} $\checkmark$
    \item[$\bullet$] per AC $\rightarrow$ D si ha (AC)\textsuperscript{+G} = ACD, D $\subseteq$ (AC)\textsuperscript{+G} $\checkmark$
    \item[$\bullet$] per E $\rightarrow$ AD si ha (E)\textsuperscript{+G} = EACDH, AD $\subseteq$ (E)\textsuperscript{+G} $\checkmark$
    \item[$\bullet$] per E $\rightarrow$ H si ha H $\subseteq$ (AC)\textsuperscript{+G} $\checkmark$
\end{itemize}
\textit{The viceversa is left as an exercise for the reader.}\par
\Huge\centering QUINDI...
\end{frame}

\subsection{Big badum-ts! time}
\begin{frame}{Equivalenza con chiusura di attributi}
    \huge
    F, G equivalenti sse...
    \begin{itemize}
        \centering
        \item[$\blacktriangleright$] $\forall$ \textbf{X} $\rightarrow$ \textbf{Y} $\in$ F, \textbf{Y} $\in$ (\textbf{X})\textsuperscript{+G}
        \item[$\blacktriangleright$] $\forall$ \textbf{X} $\rightarrow$ \textbf{Y} $\in$ F, \textbf{Y} $\in$ (\textbf{X})\textsuperscript{+G}
    \end{itemize}
\end{frame}

\begin{frame}{Conclusioni sulla chiusura di attributi}
    Sia \textbf{R}(\textbf{Z}) con le sue dipendenze in F. \par Allora, la chiusura di \textbf{X} $\subseteq$ \textbf{Z} utile per \textcolor{gray}{verificare se}:
    \begin{itemize}
        \item[$\blacktriangleright$] una dipendenza funzionale è logicamente implicata da F (vedi Teorema)
        \item[$\blacktriangleright$] un insieme di attributi è superchiave o chiave
        \begin{itemize}
            \item[$\bullet$] \textbf{X} è superchiave di \textbf{R} sse \textbf{X} $\rightarrow$ \textbf{Z} $\in$ F\textsuperscript{+}, ovvero sse \textbf{Z} $\subseteq$ (\textbf{X})\textsuperscript{+}
            \item[$\bullet$] \textbf{X} è chiave di \textbf{R} sse \textbf{X} $\rightarrow$ \textbf{Z} $\in$ F\textsuperscript{+} e $\nexists$ \textbf{Y} \colorbox{lightgray}{$\subset$} \textbf{X} tale che \textbf{Z} $\subseteq$ (\textbf{Y})\textsuperscript{+}
        \end{itemize}
    \end{itemize}
\end{frame}


% ------------------------------------------ RIDONDANZA DI UN INSIEME DI DF ------------------------------------------ %
\section{Ridondanza di un insieme di DF}

\subsection{Il Problema part.2}
\begin{frame}{Il Problema part.2}
    Si vuole usare il concetto di equivalenza tra insiemi di DF per partire da una F più semplice possibile.
    \begin{itemize}
        \item[$\blacktriangleright$] a \textcolor{blue}{destra}:
        \{A $\rightarrow$ BC\} equivalente a \{A $\rightarrow$ B, A $\rightarrow$ C\} "\textcolor{blue}{DF semplici}"
        \item[$\blacktriangleright$] a \textcolor{red}{sinistra}:
        \{A $\rightarrow$ B, AB $\rightarrow$ C\} equivalente a \{A $\rightarrow$ B, A $\rightarrow$ C\} \par
        \hspace{7cm} "senza \textcolor{red}{attributi estranei}"
        \item possono esserci \underline{DF ridondanti}, aka ottenibili da altre DF:\\
        A $\rightarrow$ C ridondante \{A $\rightarrow$ B, B $\rightarrow$ C, A $\rightarrow$ C\}
    \end{itemize}
    \vfill
    \begin{block}{}
    La riduzione della complessità può riguardare il numero di attributi che si usano in una dipendenza, o il numero di dipendenze nell'insieme.
    \end{block}
\end{frame}

\subsection{Semplificare l'insieme F}
\begin{frame}{DF semplici}
    Possiamo portare un insieme di DF F in forma “standard”, quella in cui sulla destra c’è un singolo attributo.\par
    \vfill
    Per esempio:\par
    Se F = \{AB $\rightarrow$ CD, AC $\rightarrow$ DE\} lo si scompone in:\par
            \hspace{1cm}  \{AB $\rightarrow$ C, AB $\rightarrow$ D, AC $\rightarrow$ D, AC $\rightarrow$ E\} 
    
\end{frame}

\begin{frame}{Attributi estranei}
    \begin{block}{Definizione}
    Gli attributi a sinistra di una DF che sono "inutili" (cioè la DF vale anche senza di loro) rispetto ad un dato F si dicono estranei.
    \end{block}
    Sia F = \{AB $\rightarrow$ C, A $\rightarrow$ B\}, e calcoliamo A\textsuperscript{+}, B\textsuperscript{+}:\par
    \begin{itemize}
        \item[$\bullet$] A\textsuperscript{+} = A
        \item[$\bullet$] A\textsuperscript{+} = AB poiché A $\rightarrow$ B e A $\subseteq$ A\textsuperscript{+}
        \item[$\bullet$] A\textsuperscript{+} = ABC poiché AB $\rightarrow$ C e AB $\subseteq$ A\textsuperscript{+}
    \end{itemize}
    Quindi C dipende solo da A. Per quello possiamo riscrivere F come:\par
    \begin{center}
      F := \{A $\rightarrow$ C, A $\rightarrow$ B\}
    \end{center}
    \begin{block}{}
      In generale, in una DF del tipo A\textbf{X} $\rightarrow$ B, A è estraneo se B $\subseteq$ \textbf{X}\textsuperscript{+}.
    \end{block}
    
\end{frame}

\begin{frame}{DF ridondanti}
    Dopo aver portato le DF in F in "\textcolor{cadet}{forma standard}" (o "semplici", con un solo attributo a destra) ed \textcolor{cadet}{eliminato gli attributi estranei}, vogliamo assicurarci che tutte le ridondanze in F sono necessarie.
    \begin{itemize}
        \item[$\blacktriangleright$] Come si fa a dire che \textbf{X} $\rightarrow$ A è ridondante?
    \end{itemize}
    
    \vfill
    \begin{alertblock}{Procedimento:}
        \begin{itemize}
            \item[$\bullet$] elimina \textbf{X} $\rightarrow$ A da F
            \item[$\bullet$] calcola \textbf{X}\textsuperscript{+} rispetto al nuovo F
            \item[$\bullet$] verifica se A $\in$ \textbf{X}\textsuperscript{+}
        \end{itemize}
        Se cosi, si riesce a trovare A anche senza la DF \textbf{X} $\rightarrow$ A, e quindi quest'ultima e ridondante.
    \end{alertblock}
\end{frame}

\subsection{Copertura minimale}
\begin{frame}{Copertura minimale}
    \textit{Let's pretend for a second this doesn't sound like politics.}
    \begin{block}{Definizione}
    Un insieme F di DF si dice \textcolor{blue}{copertura minimale} se:
    \begin{enumerate}
        \item[I.] è in forma standard (o semplice) --- a destra delle DF
        \item[II.] non presenta attributi estranei --- a sinistra delle DF
        \item[III.] non vi sono ridondanze
    \end{enumerate}
    In pratica, è un insieme di DF equivalente a F ma di complessità minore.
    \end{block}
    \vfill
    In generale una copertura minimale \textbf{non è unica}.
\end{frame}

\begin{frame}[fragile]{Algoritmo per trovare la Copertura Minimale}
    \begin{verbatim}
    InsiemeDF M = F
    
    per ogni ogni DF X->{A1, ..., An} in M
        sostituisci  con X->A1, ..., A->An
        
    per ogni DF X->A in M
        se A incluso in (X-{B})+
            sostituisci con (X-{B})->A
        
    per ogni DF X->A in M
        se A incluso in X+ anche rispetto a M-{X->A}
            rimuovi X->A da M
    \end{verbatim}
\end{frame}

\begin{frame}{Esempio}
Sia F = \{AB $\rightarrow$ C, B $\rightarrow$ A, C $\rightarrow$ A,\}\par
\begin{itemize}
    \item[$\blacktriangleright$] A\textsuperscript{+} = A, B\textsuperscript{+} = ABC quindi A estraneo in AB $\rightarrow$ C quindi:\\
    F' := \{B $\rightarrow$ C, B $\rightarrow$ A, C $\rightarrow$ A,\}
    \item[$\blacktriangleright$] eliminando B $\rightarrow$ C non possiamo arrivare a C, ma eliminando B $\rightarrow$ A rimane B\textsuperscript{+} = ABC per cui:\\
    F" := \{B $\rightarrow$ C, C $\rightarrow$ A,\}
\end{itemize}

\begin{alertblock}{Nota Bene}
    Tentare ad eliminare le DF ridondanti prima di risolvere gli attributi estranei risulta in inevitabile fallimento e dolore. Il dottore avvisa contro.
\end{alertblock}
\end{frame}

\begin{frame}{Esempio}
Sia F = \{A $\rightarrow$ BC, B $\rightarrow$ C, A $\rightarrow$ B, AB $\rightarrow$ C\}
\begin{itemize}
    % passo 1
    \item[$\blacktriangleright$] Portando le DF in forma standard:\\
    F := \{A $\rightarrow$ B, A $\rightarrow$ C, B $\rightarrow$ C, A $\rightarrow$ B, AB $\rightarrow$ C\}
    % passo 2
    \item[$\blacktriangleright$] Vogliamo poi eliminare possibili attributi estranei. L'unica DF che potrebbe presentarli è l'ultima, quindi si calcolano A\textsuperscript{+} = ABC, B\textsuperscript{+} = BC. Quindi sia da A che da B si può raggiungere C. Scegliamo  di togliere A, \\
    F = \{A $\rightarrow$ B, A $\rightarrow$ C, B $\rightarrow$ C, A $\rightarrow$ B, B $\rightarrow$ C\}
    % passo 3
    \item[$\blacktriangleright$] B $\rightarrow$ C e A $\rightarrow$ B appaiono due volte e quindi le possono togliere. Inoltre, calcolando A\textsuperscript{+} rispetto ad F - A $\rightarrow$ C, abbiamo sempre che A\textsuperscript{+} = ABC, e dunque la A $\rightarrow$ C ridondante.\\
    
    F = \{A $\rightarrow$ B, B $\rightarrow$ C\}
\end{itemize}
\end{frame}


\end{document}
